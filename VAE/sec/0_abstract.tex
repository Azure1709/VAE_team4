\begin{abstract}
Variational Autoencoders (VAEs) are a powerful probabilistic framework that bridges mathematical modeling and deep learning. This paper delves into the mathematical foundations of VAEs, with a focus on deriving the Evidence Lower Bound (ELBO) as a core optimization objective. Key mathematical constructs such as the Kullback-Leibler divergence and the reparameterization trick are analyzed, enabling efficient gradient computation and stable training. Additionally, we provide rigorous insights into how latent space modeling with Gaussian distributions supports data generation and reconstruction tasks. The study highlights the interplay between mathematical rigor and computational efficiency, making VAEs an essential tool in dimensionality reduction, anomaly detection, and generative modeling. This work aims to deepen the understanding of VAEs' mathematical principles, offering a strong foundation for further exploration in both academic and applied contexts.
\end{abstract}